\chapter{Conclusion} \label{ch:conclusion}
% INTRO TO SUMMARY
% Summarize the point of the thesis.
% - showed the makeup of the platform
% - nav
% - crawl
In this thesis, navigation and gaiting which are fundamental issues surrounding
mobile humanoid robotics were explored.
% We also talked about the platform that was used to facilitate these
% experiments.
Chapter~\ref{ch:platform} reviewed the platform, which detailed the major 
components and the design decisions behind the selections.
% The addition of the Lidar allowed the exploration of its use
% with humanoid robots using a stateless navigation algorithm.
The navigation algorithm was detailed in Chapter~\ref{ch:navigation}and the
experimental results were discussed in Chapter~\ref{ch:results_navigation}.
The algorithm is a lightweight stateless local planner, which performs well when
coupled with sensors that can give environmental obstacle data, such as Lidar.
% The Nao's humanoid form allowed the exploration of the crawling gait.
The gaiting methodology allowed a humanoid with a large number of
degrees-of-freedom to produce an optimized crawl gait using only three
parameters. Chapter~\ref{ch:crawl_gait} explained the method, while
Chapter~\ref{ch:results_crawl_gait} showed the results of the implementation.
% Chapter summary.
Here, the main ideas and issues will be briefly discussed, and ideas for
improvement proposed.

\section{Platform}
% Nao was a good platform for exploring crawl gaits, with some limitations.
The platform used consisted of a Nao H25 Humanoid, Hokuyo URG-04LX-UG01 Lidar, 
and a 3D printed mount to join the two. With regards to the navigation 
experiment, the Nao provided platform locomotion, onboard processing, and goal 
localization via the camera in the head. The Lidar provided range data about
obstacles in the environment. While the range was limited to 5 meters, this
was more than adequate as obstacle avoidance is a local schema. What is more
problematic is that the Lidar only provides a planar view of the environment,
and obstacles above or below this plane are not detectable.
This means that the robot could inadvertently collide with them and damage
itself or the environment.

For the crawling experiment, the Nao was used without the Lidar and was used 
only for locomotion. It was a satisfactory humanoid configuration for static
crawling experiments, though it was limited to a certain set of crawling
gaits. What was most limiting is that the joint motors had insufficient
torque to use only the arms or legs to pull/push the robot forward requiring
arm-leg cooperation to perform a crawl. Furthermore, the gearboxes in each
joint were not designed to take such loads, and frequently broke, requiring
the robot to be sent in for maintenance.
% The Lidar was good for the navigation in terms of providing planar obstacle
% avoidance. It wouldn't work for everything, as the world is 3D.

\section{Navigation}
% Summarize what was trying to be shown with the navigation chapter.
% Navigation is one of the most fundamental challenges to the area of mobile 
% robotics. 
Navigation is a core issue for mobile robots. Without the ability to navigate
safely in an environment, the robot can damage both itself and its surroundings.
% Navigation needs a layered approach. Here we explored a local approach. 
Navigation requires a layered approach, with global path planning typically
utilizing different schemas to local navigation and obstacle avoidance.
GODZILA is a local approach, based on the potential field idea.
With the Lidar providing rich environmental data, GODZILA was able to guide
the robot around different obstacle configurations. It was also computationally
fast and required a small memory footprint, as it does not build a map.
While it was able to successfully navigate in the environments shown, potential
field approaches have some short-comings. Narrow corridors can present a problem
as tuning the algorithm to make them traversable, typically also brings the
robot too close to obstacles in other scenarios. Escaping traps is also an
issue. Theoretically, this issue can again be overcome by appropriate tuning,
but it will cause the performance to suffer in other areas.
Ultimately, both of these scenarios require more detailed analysis of the
environment to resolve, recover, or avoid these problems and bias parameters
or produce intermediate goal points. This is why the navigation problem is
typically approached using many algorithms, layered together.

\section{Crawl Gait}
% Summarize what was trying to be shown with the crawling chapter.
Different modes of locomotion allow the robot to navigate through a larger set
of environments. Walking gaits are important to humanoid locomotion, but are
not appropriate in all scenarios.
% Crawling is something that humanoids do quite often, so it's
% a good thing to explore.
Crawling is more stable than walking, and allows a humanoid to traverse
beneath low overhangs and small apertures. This extends the navigable area of
and environment.
% Talk about how we were able to show these things with the Nao.
Using the Projected Profile approach, the Nao was able to crawl under objects
hanging 8 inches above the ground. When standing, the Nao is 23 inches tall.
% So, we were able to devise a simple crawl gait for humanoids
% (or really any robot like this), which we showed on the Nao.
In addition, the crawl gait is able to parameterize a 25 degree-of-freedom
system using only 3 parameters. By modifying these parameters, the robot can
devise an energy efficient gait, or control the pose of the back throughout
the gait.
One issue with the current implementation, is that there are no parameters
available to control the direction of travel. Therefore, the robot can only
crawl forward in an open loop fashion, unable to correct its trajectory if it
starts to veer off course. Such a low profile, also prevents the Nao from using
the Lidar to sense the environment. This is because the mount designed for the
Lidar, puts it at the waist of the robot which will interfere with the gait.

\section{Future Work}
% While things worked, there are things that could be improved on.
The presented work provides a foundation to expand the capabilities for
mobile humanoid robots. Using this, several improvements are planned to both
improve the experimental platform and the algorithms used to be more effective.
% IMPROVEMENTS TO PLATFORM
% Lidar prevents crawling, but high up prevents walking.
The addition of the Lidar to the platform greatly expands the obstacle avoidance
capabilities of the robot, as well as allowing the use of commonly available
Lidar based simultaneous localization and mapping algorithms.
An issue with adding such a device is positioning it on the Nao humanoid.
Placing the Lidar at hip level is a good choice for navigation as it is one of
the parts of the robot that moves the least, but makes locomotion modalities
such as crawling problematic. A better choice might be mounting it on the head,
as it will not interfere with crawling, and can be oriented using the neck.
This could also allow the robot to choose the scan plane during locomotion
and might allow obstacle detecting during crawling.
The major obstacle to this path is the lack of walking gait parameters to be 
tuned using the NAOqi API. The effect is, due to the walking gait not knowing
about the additional mass and change in inertia, the gait is destabilized.
When used with the current Lidar mount, this caused the robot to fall over
during navigation. When experimenting with mounting the Lidar on the head,
the problem was magnified and, in general, prevented the robot from walking
any significant distance.
Two options are available to overcome this challenge.
This first is to devise a new walking gait, where these unmodeled parameters
can be accounted for. A second approach is to command the arms of the robot
to counteract the instability. This is potentially an easier approach than
devising a new walking gait. It could also produce a new line of research into
not only improving walking gaits, but reducing the occurrence of humanoids
falling over when encountering unanticipated hazards. 
An additional platform enhancement would be to use a 3D Lidar.
This would allow the robot to detect obstacles in the entire collision space
of the robot and make navigation safer.
Finally, as the Nao is equipped with cameras in the head, as vision-based
approach to obstacle detection is also possible. As the robot is not equipped
with a stereo pair, stereo vision techniques are infeasible. Despite this,
a set of related algorithms called structure-from-motion are compatible with
monocular-camera configurations.
% Walking gait needs to be improved in order to account for the additional mass
% in a different place. 
% One idea is to swing the arms in a way to stabilize the platform.

% 3D Lidar.

% IMPROVEMENTS TO NAVIGATION
% Stuck detection is always a bitch, and almost cannot be done memoryless,
% by definition. Trap avoidance is also hard. Likely, some of those other layers
% will need to be implemented to avoid these issues.
While the benefits of the GODZILA technique have been presented, extensions
are needed to form a robust navigation solution in complicated environments.
Memoryless algorithms are unable to handle all trap conditions, and a system
to detect and recover or avoid these scenarios is necessary. Some form of map
is needed so the scenario can be analyzed. A local sliding window map that the
robot can localize itself within, could be used to detect if the
Nao hasn't moved from a region within a reasonable time and form a new plan
or mark the trap as a virtual obstacle. A global map would inform
the robot if it hasn't made significant progress toward the goal or facilitate 
the planning of a path that avoids many traps. Each of these approaches comes 
at the cost of increased memory footprint and computational complexity.

% IMPROVEMENTS TO CRAWLING
% Need to turn. Probably modulate the arm and leg cycles.
The current Projected Profile scheme does not reason about the robot turning.
This will have to be addressed in order to make it more robust and widely 
applicable as the ability to correct the robot's path is a necessary feature.
% There are also energy efficient things to tune.
The presented implementation of the Projected Profile algorithm was optimized
to minimize static joint torques. An extension to this is to optimize for
dynamic torques by having the optimization routine choose a parameter set and 
test the resultant gait with the simulator. Additionally, instead of having
a single parameter set utilized by the robot, an entire library of parameters
could be produced for use by the robot. This would allow the robot to choose
which parameters to use based on other gaiting criteria such as speed, head
stability, or trajectory controllability. 
% Also, we could have other goals, like back stabiliztion to keep the lidar
% in the right orientation.
% Use the camera to avoid obstacles.
% Try other terrains, use the IMU.
IMU's have become a common addition to mobile robots and is a standard
feature on the Nao humanoids. The IMU on the Nao, could be used to inform the
pose of the torso and infer if the robot is on a sloping terrain. The robot
could then adjust its posture during gaiting to more aggressively crawl up
a hill, or be more cautious when crawling into a pit.

% FUTURE WORK CONCLUSION
There are many areas using the current platform that can be improved to produce
a more robust set of solutions for use with humanoid robots. We look forward
to exploring these in the future and are excited for the possibilities in the
field.
